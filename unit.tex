\documentclass[]{beamer}
\usepackage[utf8]{inputenc}
\usepackage[english]{babel}
\usepackage{hyperref}
\usepackage{ulem}\normalem
\usepackage{textcomp}

\newcommand{\fondoTitulo}{%
  \usebackgroundtemplate{\includegraphics[width=\paperwidth]{fondo}}}
\newcommand{\fondoTexto}{%
  \usebackgroundtemplate{\includegraphics[width=\paperwidth]{fondo}}}

\definecolor{celeste}{HTML}{5E91AA}
\definecolor{azul}{HTML}{163F54}

\setbeamercolor{head1}{fg=celeste}
\setbeamercolor{title}{fg=celeste}
\setbeamercolor{subtitle}{fg=celeste}
\setbeamercolor{frametitle}{fg=celeste}
\setbeamercolor{structure}{fg=azul}
\setbeamercolor{normal text}{fg=azul}

\title{\textbf{\LARGE{Mi titulo de charla}}}
\author[Expositor]{Expositor}
\institute[Institución]{}
\date[]{Fecha}

\AtBeginSection[]
{
  \begin{frame}
    \frametitle{Programa}
    \tableofcontents[currentsection]
  \end{frame}
}

\begin{document}

\fondoTitulo
\begin{frame}
  \titlepage
\end{frame}

\fondoTexto
\section{Antecedentes}
\begin{frame}{Titulo}
  \begin{itemize}
  \item Mi info
  \end{itemize} 
\end{frame}

\begin{frame}{Mockito vs Jmock}
  \begin{itemize}
  \item Jmock usa un contexto para manejar los mocks
  \item Mockito lo hace con metodos importados estáticamente
  \end{itemize} 
\end{frame}

\begin{frame}{Mockito vs Jmock}
  \begin{itemize}
  \item Jmock esta diseñado para mockear interfaces, no clases
  \item Mockito lo hace con metodos importados estáticamente
  \item Mockito utiliza multiples mocks para la misma clase por medio de identificadores
  \item Mockito soporta anotaciones
  \item EL ORDEN EN MOCKITO ES IMPORTANTE/ISIMO
  \item Verify siempre viene despues
  \item mockito mas facil de leer
    \item section matchers
    \item uso de any
    \item ambos tienen problemas combinando
  \item numero de invocaciones
    \item mockito usa verify
    \item mockito no revisa si se hace una llamada no esperada
    \item jmock tiene allowing e ignoring
  \item call order
  \item jmock usa sequence
    \item mockito usa inOrder
  \end{itemize} 
\end{frame}


\fondoTitulo
\begin{frame}
  \begin{center}
   \LARGE ¡Gracias!   \\
   Mi nombre
    \texttt{mi correo} 
    \url{mi pagina}
  \end{center}    
\end{frame}

\end{document}
