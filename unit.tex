\documentclass[]{beamer}
\usepackage[utf8]{inputenc}
\usepackage[english]{babel}
\usepackage{hyperref}
\usepackage{ulem}\normalem
\usepackage{textcomp}
\usepackage{color}

\newcommand{\fondoTitulo}{%
  \usebackgroundtemplate{\includegraphics[width=\paperwidth]{fondo}}}
\newcommand{\fondoTexto}{%
  \usebackgroundtemplate{\includegraphics[width=\paperwidth]{fondo}}}

\definecolor{celeste}{HTML}{5E91AA}
\definecolor{azul}{HTML}{163F54}
\definecolor{negro}{HTML}{324085}


\setbeamercolor{head1}{fg=celeste}
\setbeamercolor{title}{fg=celeste}
\setbeamercolor{subtitle}{fg=celeste}
\setbeamercolor{frametitle}{fg=celeste}
\setbeamercolor{structure}{fg=azul}
\setbeamercolor{normal text}{fg=negro}
\setbeamercolor{verbatim}{fg=negro}

\title[jMock->Mockito]{\textbf{\LARGE{Migrando de jMock a Mockito}}}
\subtitle[]{Evoluciona o muere}
\author[aranax]{Wences Arana}
\institute[lugusac, debian-gt, guatejug]{lugusac, debian-gt, guatejug}
\date[14/11/2015]{14 de noviembre de 2015}

\AtBeginSection[]
{
  \begin{frame}
    \frametitle{Programa}
    \tableofcontents[currentsection]
  \end{frame}
}

\AtBeginSubsection[]
{
  \begin{frame}
    \frametitle{Programa}
    \tableofcontents[currentsection,currentsubsection]
  \end{frame}
}

\makeatletter
\def\verbatim{\tiny\@verbatim \frenchspacing\@vobeyspaces \@xverbatim}
\makeatother

\begin{document}

\fondoTitulo
\begin{frame}
  \titlepage
\end{frame}

\fondoTexto
\transboxout
\section{Frameworks}
\subsection{jMock}
\begin{frame}[c]
  \frametitle{jMock}
  \textbf{Features:}\footnote{http://www.jmock.org/index.html}
  \pause
  \begin{itemize}
    \pause \item Provee una extensa librería de mocking objects
    \pause \item Fácil y rápido
    \pause \item Librería extensa
    \pause \item Completamente declarativa
    \pause \item Fácil de extender
    \pause \item expect-run-verify
  \end{itemize}
\end{frame}

\subsection{Mockito}
\begin{frame}
  \frametitle{Mockito}
  \textbf{Features:}\footnote{https://code.google.com/p/mockito/wiki/FeaturesAndMotivations}
  \begin{itemize}
  \pause \item Desarrollo continuo (push)
  \pause \item JDK8 compatible (Defenders)
  \pause \item API limpia y simple
  \pause \item Fácil lectura
  \pause \item Bastante utilizada
  \end{itemize}
\end{frame}

\begin{frame}
  \frametitle{Mockito}
  \textbf{Features(continuación...)}
  \begin{itemize}
  \pause \item Mockea clasess además de interfaces
  \pause \item ANNOTATIONS o/ 
  \pause \item Stack trace limpio 
  \end{itemize}
\end{frame}

\begin{frame}[c]
  \frametitle{Funcionamiento}
  \begin{itemize}
  \pause \item \alert{No es una librería expect-run-verify}
  \pause \item \alert{NO EXISTEN LAS EXPECTATIONS}
  \pause \item Preguntar sobre interacciones despues de ejecutar
  \pause \item Verifica solo lo que necesites (I can test what I want)
  \pause \item Stubbing va antes de la ejecución, verificaciones de interacción después
  \end{itemize}
\end{frame}

\section{Diferencias entre jMock y Mockito}
\subsection{Creando mocks}
\begin{frame}
\frametitle{Mockito vs jMock}
  \begin{itemize}
  \item jMock usa un contexto para manejar los mocks
  \item mockito lo hace con metodos importados estáticamente
  \item jMock esta diseñado para mockear interfaces, no clases
  \item mockito puede mockear clases
  \item mockito soporta anotaciones
  \end{itemize}
\end{frame}

\begin{frame}[fragile]
\frametitle{Mockito vs jMock}\footnote{http://zsoltfabok.com/blog/2010/08/jmock-versus-mockito/}
\textbf{jMock}
\begin{verbatim}
import org.jmock.Mockery;
import org.junit.Test;

public class PosTerminalTest {
    @Test
    public void shouldPrintSuccessfulWithdrawal() {
        Mockery context = new Mockery();

        BankConnection bankConnection = context.mock(BankConnection.class);
        ReceiptPrinter receiptPrinter = context.mock(ReceiptPrinter.class);
        // the test case continues...
    }
}  
\end{verbatim}
\textbf{Mockito}
\begin{verbatim}{
import static org.mockito.Mockito.mock;
import org.junit.Test;

public class PosTerminalTest {
    @Test
    public void shouldPrintSuccessfulWithdrawal() {
        BankConnection bankConnection = mock(BankConnection.class);
        ReceiptPrinter receiptPrinter = mock(ReceiptPrinter.class);
        // the test case continues...
   }
}
\end{verbatim}
\end{frame}

\begin{frame}[fragile]
\textbf{Mockito(continuación ...)}
\begin{verbatim}
import org.mockito.Mock;
// Other import statements ...

@RunWith(MockitoJUnitRunner.class)
public class PosTerminalTest {
    @Mock
    private BankConnection bankConnection;
    @Mock
    private ReceiptPrinter receiptPrinter;
    @Mock
    private Account customerAccount;
    @Mock
    private Account shopAccount;

    @Test
    public void shouldPrintSuccessfulWithdrawal() {
    }
\end{verbatim}
\end{frame}

\subsection{Verificación de llamadas y valores de retorno}
\begin{frame}
  \begin{itemize}
    \item mockito usa verify
    \item mockito no revisa si se hace una llamada no esperada
    \item jMock tiene allowing e ignoring
  \end{itemize}
\end{frame}

\begin{frame}[fragile]
\textbf{jMock}
\begin{verbatim}
@Test
    public void shouldPrintSuccessfulWithdrawal() {
        PosTerminal posTerminal = new PosTerminal(bankConnection, receiptPrinter);

        context.checking(new Expectations() { {
            oneOf(bankConnection).getAccountByCardNumber("1000-0000-0001-0003");
            will(returnValue(customerAccount));

            // Other oneOf() will() statements ...

            oneOf(receiptPrinter).print("Successful withdrawal");
        } });

        posTerminal.buyWithCard(100, "1000-0000-0001-0003");
    }
\end{verbatim}
\textbf{mockito}
\begin{verbatim}
@Test
    public void shouldPrintSuccessfulWithdrawal() {
        PosTerminal posTerminal = new PosTerminal(bankConnection, receiptPrinter);

        when(bankConnection.getAccountByCardNumber("1000-0000-0001-0003"))
            .thenReturn(customerAccount);

        // Other when(...).thenReturn(...) calls ...

        posTerminal.buyWithCard(100, "1000-0000-0001-0003");

        verify(receiptPrinter).print("Successful withdrawal");
    }
\end{verbatim}
\end{frame}

\subsection{Argument matchers}
\begin{frame}
  \begin{itemize}
    \item Uso de any
    \item Ambos tienen problemas combinando
  \end{itemize}
\end{frame}

\begin{frame}[fragile]
\textbf{jMock}
\begin{verbatim}
context.checking(new Expectations() { {
            // Other oneOf() will() statements ...

            oneOf(customerAccount).withdraw(with(any(Integer.class)),
                    with(any(String.class)));
            will(returnValue(true));

            oneOf(shopAccount).enter(with(100), with(any(String.class)));
            will(returnValue(true));

           // Other oneOf() will() statements ...
        } });
\end{verbatim}
\textbf{mockito}
\begin{verbatim}
import static org.mockito.Matchers.anyInt;
import static org.mockito.Matchers.anyString;
import static org.mockito.Matchers.eq;
// Other parts of the code ...
    @Test
    public void shouldPrintSuccessfulWithdrawal() {
       // Other when(...).thenReturn(...) calls ...

        when(customerAccount.withdraw(anyInt(), anyString())).thenReturn(true);
        when(shopAccount.enter(eq(100), anyString())).thenReturn(true);

        // Other when(...).thenReturn(...)  and verify(...) calls ...
    }
\end{verbatim}  
\end{frame}

\subsection{Number of invocations}
\begin{frame}[fragile]
\textbf{jMock}
\begin{verbatim}
context.checking(new Expectations() { {
           // Other oneOf() will() statements ...
            exactly(1).of(receiptPrinter).print(with(any(String.class)));
        } });
or
context.checking(new Expectations() { {
           // Other oneOf() will() statements ...
            atLeast(1).of(receiptPrinter).print(with(any(String.class)));
        } });
or
context.checking(new Expectations() { {
           // Other oneOf() will() statements ...
            never(receiptPrinter).print(with(any(String.class)));
        } });
\end{verbatim}
\end{frame}

\begin{frame}[fragile]
\textbf{mockito}
\begin{verbatim}
// Other when(...).thenReturn(...) calls ...
        verify(receiptPrinter, times(1)).print(anyString());
        // Other verify() calls ...
or
// Other when(...).thenReturn(...) calls ...
        verify(receiptPrinter, atLeast(1)).print(anyString());
        // Other verify() calls ...
or
// Other when(...).thenReturn(...) calls ...
        verify(receiptPrinter, never()).print(anyString());
        // Other verify() calls ...
\end{verbatim}
\end{frame}

\subsection{Llamadas consecutivas}
\begin{frame}[fragile]
\textbf{jMock}
\begin{verbatim}
context.checking(new Expectations() { {
            // Other oneOf() will() statements ...

            oneOf(shopAccount).enter(with(100), with(any(String.class)));
            will(onConsecutiveCalls(returnValue(true), returnValue(false),
                 returnValue(false)));
        } });
\end{verbatim}
\textbf{mockito}
\begin{verbatim}
// Other when(...).thenReturn(...) calls ...
        when(shopAccount.enter(eq(100), anyString())).thenReturn(true, false, false);
\end{verbatim}
\end{frame}

\subsection{Verificación de orden de llamadas}
\begin{frame}{Mockito vs Jmock}
  \begin{itemize}
  \item jMock usa sequence
  \item mockito usa inOrder
  \end{itemize}
\end{frame}


\begin{frame}[fragile]
\textbf{jMock}
\begin{verbatim}
@Test
    public void shouldPrintSuccessfulWithdrawal() {
        PosTerminal posTerminal = new PosTerminal(bankConnection, receiptPrinter);

        final Sequence sequence = context.sequence("account order");
        context.checking(new Expectations() { {
            // Other oneOf() will() statements ...

            oneOf(customerAccount).withdraw(with(any(Integer.class)),
                    with(any(String.class)));
            will(returnValue(true));
            inSequence(sequence);

            atLeast(1).of(shopAccount).enter(with(100), with(any(String.class)));
            will(returnValue(true));
            inSequence(sequence);
        } });

        posTerminal.buyWithCard(100, "1000-0000-0001-0003");
    }
\end{verbatim}
\end{frame}

\begin{frame}[fragile]
  \textbf{mockito}
\begin{verbatim}
@Test
    public void shouldPrintSuccessfulWithdrawal() {
        PosTerminal posTerminal = new PosTerminal(bankConnection, receiptPrinter);

        posTerminal.buyWithCard(100, "1000-0000-0001-0003");

        InOrder inOrder = Mockito.inOrder(customerAccount, shopAccount);
        inOrder.verify(customerAccount).withdraw(anyInt(), anyString());
        inOrder.verify(shopAccount).enter(anyInt(), anyString());

        verify(receiptPrinter).print(anyString());
    }
\end{verbatim}
\end{frame}

\subsection{Manejo de excepciones}
\begin{frame}[fragile]
\textbf{jMock}
\begin{verbatim}
context.checking(new Expectations() { {
            oneOf(bankConnection).getAccountByCardNumber("1000-0000-0001-0003");
            will(throwException(new RuntimeException()));
         } });
\end{verbatim}
\textbf{mockito}
\begin{verbatim}
when(bankConnection.getAccountByCardNumber("1000-0000-0001-0003"))
          .thenThrow(new RuntimeException());
\end{verbatim}
\end{frame}

%team

%game

%Review state verification _after the tests have run_
%Behavior verification _checking to see if certain methods where invoked(matchers, etc)_

%JMOCk: set up expectations, then execute and finally verify
%Mockito verify mock invocation after the test is executed

%Ver http://blog.trafficparrot.com/2014/05/do-you-have-problems-with-your-test.html
%https://www.linkedin.com/pulse/what-difference-between-stub-mock-virtual-service-wojciech-bulaty

%Spaceballs
%https://www.linkedin.com/pulse/what-difference-between-stub-mock-virtual-service-wojciech-bulaty

%http://www.vogella.com/tutorials/Mockito/article.html

%ohttp://jmockit.org/MockingToolkitComparisonMatrix.html

%http://blogs.justenougharchitecture.com/unit-testing-with-mocks-easymock-jmock-and-mockito/


%% \fondoTitulo
\begin{frame}[c]
   \begin{center}
     \LARGE DEMO
   \end{center}    
\end{frame}

\begin{frame}[c]
   \begin{center}
     \LARGE ¡Gracias!
   \end{center}    
\end{frame}

\end{document}
